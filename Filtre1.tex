\begin{verbatim}
1  -	%Récupération du son
2  -	
3  -	%Définition de la période d'échantillonnage
4  -	fs=44100;
5  -	%Définition du nombre de voies d'enregistrement
6  -	noc=1;
7  -	%Définition du nombre de bits
8  -	nob=16;
9  -	%Enregistrement du signal audio
10 -	recObj=audiorecorder(fs,nob,noc);
11 -	record(recObj);
12 -	pause(40);
13 -	stop(recObj);
14 -	%Lancement du signal enregistré
15 -	play(recObj);
16 -	%Formatage du signal enregistré sous forme vectoriel
17 -	myrecordings=getaudiodata(recObj);
18 -	%Représentation du signal enregistré
19 -	plot(myrecordings);
20 -	%Fin de la partie prise de son
21 -	
22 -	%Début de l'algorithme et lancement du chronométrage
23 -	tic;
24 -	%Initialisations et déclaration des constantes
25 -	
26 -	yk=ones(1,64);
27 -	%Déclaration des constantes "beta", "delta", "rho" et "mu".
28 -	b=100;
29 -	D=(1/64);
30 -	rho=0.01;
31 -	mu=0.8;
32 -	%Initialisation des W pour NLMS puis PNLMS.
33 -	w1=(ones(1,64))';
34 -	w2=(ones(1,64))';
35 -	%Initialisation des vecteurs utiles au calcul de Gk.
36 -	gam=zeros(1,64);
37 -	gk=zeros(1,64);
38 -	%Initialisation des vecteurs d'erreur et de W.
39 -	err=(1:64000);
40 -	w=zeros(1:64);
41 -	
42 -	%Début du traitement
43 -	for i=1:20850
44 -	    %Récupération du vecteur d'entrée à chaque étape i
45 -	    sign=(myrecordings((64+(i-1)):-1:(1+(i-1))))';
46 -	    
47 -	    %Calcul d'un paramètre utile pour trouver le vecteur Gk
48 -	    Vk=max([D,abs(w)]);
49 -	    %Calcul du vecteur Gk
50 -	    for k=1:64
51 -	        gam(k)=max(((rho)*Vk),(abs(w(k))));
52 -	    end
53 -	    for M=1:64
54 -	        gk(M)=((gam(M))/(mean(gam)));
55 -	    end
56 -	    Gk=1*diag(gk);
57 -	    %Calcul des éléments du vecteur d'erreur en gérant le début
58 -	    for s=1+(i-1):64+(i-1)
59 -	        sign2=(myrecordings((64+(s-1)):-1:(1+(s-1))))';
60 -	        yk(s) = (sign2)*w';
61 -	        err(s+63)=((yk(s)-myrecordings(64+(s-1))));
62 -	    end
63 -	    
64 -	    %Début du processus de réinitialisation du filtre
65 -	    
66 -	    %Algorithme PNLMS (Calcul du vecteur W1)
67 -	        w1=(w'+(mu*Gk*((sign)')*err(i+63))/(sign*Gk*(sign)'+b)));
68 -	    %Algorithme NLMS (Calcul du vecteur W2)    
69 -	        w2=(w'+((mu*((sign)')*err(i+63))/(((sign)*((sign)')+b))));
70 -	        
71 -	        
72 -	    %Réinitialisation des coefficients du filtre par PNLMS++
73 -	    
74 -	    for k=1:64
75 -	        %Coefficients impairs par PNLMS
76 -	        if mod(k,2)==1
77 -	          w(k)=w1(k);
78 -	        %Coefficients pairs par NLMS
79 -	        elseif mod(k,2)==0
80 -	            w(k)=w2(k);
81 -	        end
82 -	    end
83 -	end
84 -	%Fin du chronométrage
85 -	toc;
\end{verbatim}
\textbf{Revenir dans le texte en cliquant \ref{Brut}}.