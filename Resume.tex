Ce travail consiste à concevoir un filtre numérique destiné à un système annulateur d'écho sonore.\\
Pour y parvenir, nous commençons par passer en revu les notions générales de la théorie de traitement du signal directement utiles à cette fin. Après cela, une analyse détaillée des algorithmes élaborés pour l'identification des systèmes est réalisée pour finalement déboucher sur deux choses: l'étude des techniques permettant d'optimiser au maximum les algorithmes et le choix d'un al\-go\-ri\-th\-me particulièrement adapté aux systèmes annulateurs d'écho (nous avons choisi le \textbf{BPNLMS++} ). Ici le système à identifier est le chemin d'écho. L'algorithme choisi et amélioré est écrit et testé sous \textbf{Matlab} pour en vérifier la validité (le test de validation est réalisé dans une situation où l'on sait déjà le résultat prévu par la théorie et on simule l'algorithme pour le valider ou l'invalider). La validation ayant réussi, le filtre est prêt à être implémenté comme tel sur une plateforme d'annulation d'écho. Le filtre conçu est évidemment adaptatif étant donné que les signaux sonores sont non stationnaires; l'algorithme sert donc à l'adaptation des coefficients du filtre jusqu'à ce qu'il devienne utilisable dans l'enceinte considérée (en peu de temps et chaque fois que c'est nécessaire). Donc ces coefficients sont figés pendant le filtrage. Ainsi, grâce à ce filtre, nous pouvons modéliser l'écho d'un signal en entrée, et le soustraire du signal global à transmettre comme réponse à l'émetteur, pour que ce dernier ne puisse pas recevoir une réplique du signal qu'il a émis (l'écho est ainsi supprimé).\\
Le travail est effectivement clos par la validation de l'algorithme qui implémente le filtre adaptatif conçu, ainsi que par un essai d'annulation d'écho.\\
$ _{ } $\\
\textbf{\underline{Mots-clés}} : Traitement du signal, annulateur d'écho sonore, filtre adaptatif, Processeur de signaux numériques, Matlab.