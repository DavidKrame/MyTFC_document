Au début de ce travail, nous pensions devoir considérer en profondeur les phénomènes physiques sous-jacents à la propagation des ondes sonores. Au cours des recherches, nous nous sommes rendu bien compte qu'il était plus rapide et rentable de modéliser juste l'ensemble, pour pouvoir mener à bien le travail à un coût dérisoire. Il est en effet plus optimal de traiter les phénomènes complexes, comme les phénomènes de propagation du son en dehors des conditions de laboratoire, de manière approximative mais globale au lieu de les voir en détail. Comme cela a bien été souligné, l'approche détaillée est utopique vu que le coût de calcul est immensément énorme.\\
En introduction, nous avons posé comme hypothèse, que le filtrage est une solution à la présence indésirée d'écho dans un système de communication, et ce travail vient évidemment de nous conforter dans cette affirmation. Nous avons également supposé qu'un filtrage purement analogique serait moins rentable mais nous nous sommes rendu compte au fil du travail que, cette proposition doit être bien nuancée. En effet, durant ce travail, nous avons montré que nous ne pouvons, si nous considérons le système dans son ensemble (et non seulement la partie qui concerne le traitement algorithmique), nous passer d'une partie de traitement analogique du signal (penser au filtre anti-repliement par exemple). Néanmoins, en nous basant uniquement sur le filtrage du signal reçu, sans trop nous fier au formatage et à la correction du rapport signal sur bruit, nous pouvons sans doute nous fier uniquement à un traitement purement numérique. Finalement, au vu du processus mis en œuvre pour l'adaptation du filtre annulateur d'écho, nous confirmons la troisième hypothèse lancée en introduction et affirmant qu'on ne peut se passer d'une mémorisation dynamique du signal durant une partie du traitement pour pouvoir mener à bien l'annulation de l'écho sonore lié au système de communication.
\paragraph{}
Ce travail a pointé du doigt la puissance de la théorie mathématique de traitement du signal. C'est ainsi  qu'il a pu déboucher sur le filtrage numérique et en premier lieu sur le filtrage de Wiener qui est évidemment un prélude inévitable pour aborder le filtrage adaptatif.\\
Le filtrage adaptatif car, comme cela a été plusieurs fois mentionné, c'est le moyen par excellence pour pouvoir modéliser un système non stationnaire comme c'est le cas pour les systèmes sonores. Notre travail a donc consisté à élaborer un filtre numérique modélisant l'enceinte dans laquelle le son va se propager, de telle sorte qu'il ne puisse pas y avoir une grande différence entre le son perçu à la sortie du filtre et celui qui serait perçu après avoir parcouru la salle. Cela revient donc à modéliser le chemin d'écho. Pour cela, les coefficients du filtre élaboré doivent correspondre à la réponse impulsionnelle de l'enceinte (à l'endroit où l'on se tient pour la mesurer).\\
Toutefois, les signaux sonores étant très instationnaires, il est impérieux d'adapter ces coefficients au fur et à mesure de la réception du signal. D'où la nécessité des filtres adaptatifs comme on l'a tantôt montré. Nous avons développé les algorithmes du filtrage adaptatif en nous focalisant sur les plus indiqués pour l'identification des système (le chemin d'écho pour notre cas). Cela nous a conduit à considérer un algorithme nommé \textbf{PNLMS++} qui est très indiqué dans des tels systèmes pour sa grande vitesse de convergence et sa robustesse.\\
L'objectif visé est qu'en cas de besoin, les théories abordées dans ce travail puissent être exploitées directement pour élaborer des systèmes devant équiper des appareils de communication habituels pour l'annulation d'échos sonores éventuels. C'est pour cette raison qu'il nous a semblé très important de montrer de quelle manière on pourrait réduire toujours plus fortement la charge de calcul des processeurs numériques à utiliser ainsi que le coût (d'un point de vu pécuniaire) de l'ensemble. De toute évidence, les processeurs à virgule fixe sont les plus adaptés à l'élaboration des plate-formes à faible prix tout en gardant une efficacité admissible quand ils sont programmés en conséquence. Ils allient en fait la rentabilité financière à la précision en ce qui concerne notre projet. Ce qui précède justifie le choix d'un traitement tout à fait particulier des données pour les rendre adaptées le plus fortement possible aux processeurs à virgule fixe (il s'agit du traitement par transformation en nombres de Fermat en lieu et place de la transformation de Fourier). Cette nouvelle transformation a toutes les propriétés nécessaires pour développer et optimiser la majorité des processus d'annulation d'écho dès qu'on décide de les implanter sur un processeur à virgule fixe.\\
Au préalable, nous prévoyions réaliser les algorithmes en langage C mais nous étant rendu compte du fait que cela serait sans intérêt( en première approximation), nous avons choisi de développer le système sous Matlab sachant que si on envisageait une implantation sur DSP, l'adaptation du code serait simple étant donné la similarité des syntaxes de ces deux langages. C'est sous cet angle que nous avons mis au point sous Matlab un algorithme complet d'adaptation des coefficient du filtre destiné à l'annulateur d'écho.\\
Aux futurs chercheurs, nous suggérons l'étude complète de l'incorporation des algorithmes ici conçus, aux systèmes de communication complets ainsi que, la correction des erreurs introduites dans le filtre obtenu en cas de traitement par blocs, pour améliorer encore plus l'atténuation de l'écho par cette méthode vu que ce dernier traitement est le mieux adapté au fonctionnement en temps réel.
