\begin{verbatim}
1   -	%Récupération du son
2   -	
3   -	%Définition de la période d'échantillonnage en respectant Shannon
4   -	fs=44100;
5   -	%Définition du nombre de voies d'enregistrement
6   -	noc=1;
7   -	%Définition du nombre de bits
8   -	nob=16;
9   -	%Enregistrement du signal audio
10  -	recObj=audiorecorder(fs,nob,noc);
11  -	record(recObj);
12  -	pause(40);
13  -	stop(recObj);
14  -	%Lancement du signal enregistré
15  -	play(recObj);
16  -	%Formatage du signal enregistré sous forme vectoriel
17  -	myrecordings=getaudiodata(recObj);
18  -	%Représentation du signal enregistré
19  -	plot(myrecordings);
20  -	%Fin de la partie prise de son
21  -	
22  -	%Début de l'algorithme et lancement du chronométrage
23  -	tic;
24  -	%Initialisations et déclaration des constantes
25  -	
26  -	w=zeros(1,64);
27  -	zero=zeros(1,64);
28  -	%Complément de W par des zeros pour atteindre la longueur réquise
29  -	wk=[w,zero]';
30  -	%Déclaration des constantes "beta", "delta", "rho" et "mu".
31  -	b=100;
32  -	D=(1/64);
33  -	rho=(10^(-2));
34  -	mu=(0.8);
35  -	%Initialisation des W pour BNLMS puis BPNLMS.
36  -	w1=(ones(1,64))';
37  -	w2=(ones(1,64))';
38  -	%Initialisation des vecteurs utiles au calcul de Gk.
39  -	gam=zeros(1,64);
40  -	gk=zeros(1,64);
41  -	
42  -	%Début du traitement
43  -	for i=1:20850
44  -	    %Sélection particulier du premier bloc d'échantillons
45  -	    if i==1
46  -	        sign=(myrecordings(1:128))';
47  -	        sign2=(myrecordings(128:-1:1));
48  -	    %Sélection général de tous les blocs    
49  -	    elseif i>=2
50  -	        sign=(myrecordings(65+64*(i-2):192+64*(i-2)))';
51  -	        sign2=(myrecordings((192+64*(i-2)):-1:(65+64*(i-2))));
52  -	    end
53  -	    
54  -	    %Transformation du signal d'entrée et du vecteur W
55  -	    SIG=fft(sign);
56  -	    W=(fft(wk))';
57  -	    %Calcul du produit terme à terme des deux vecteurs
58  -	    Y=SIG.*W;
59  -	    %Transformation inverse pour retrouver le produit de convolution
60  -	    YW=ifft(Y);
61  -	    %Sélection des termes utiles recherchés
62  -	    yw=YW(65:128);
63  -	    %Calcul du vecteur d'erreurs
64  -	    err=(myrecordings(1+64*(i-1):64*i))'-(yw);
65  -	    %Complément du vecteur par des zeros puis transformation
66  -	    ERR=[err,zero];
67  -	    errF=fft(ERR);
68  -	    %Transformation du vecteur image
69  -	    SIG2=fft(sign2);
70  -	    %Réalisation de l'étape 3°) points 15, 16 et 17 de l'algorithme
71  -	    ERsign=(errF)'.*(SIG2);
72  -	    erSIGN=ifft(ERsign);
73  -	    es=erSIGN(65:128);
74  -	    %Réalisation de l'étape 3°) points 18 et 19 (autocorrélation)
75  -	    RSS=SIG2.*(SIG)';
76  -	    RSs=ifft(RSS);
77  -	    Rss=RSs(65:128);
78  -	    
79  -	    %Calcul d'un paramètre utile pour trouver le vecteur Gk
80  -	    Vk=max([D,abs(w)]);
81  -	    %Calcul du vecteur Gk
82  -	    for k=1:64
83  -	        gam(k)=max(((rho)*Vk),(abs(w(k))));
84  -	    end
85  -	    for M=1:64
86  -	        gk(M)=((gam(M))/(mean(gam)));
87  -	    end
88  -	    Gk=diag(gk);
89  -	    %Réalisation de l'étape 3°) point 21
90  -	    GkRss=Gk*(Rss);
91  -	    
92  -	    GkEsi2=Gk*es;
93  -	    
94  -	    %Réinitialisation des coefficients du filtre
95  -	    
96  -	    for j=1:64
97  -	        %Algorithme BPNLMS
98  -	        w1(j)=(w(j)+((mu*GkEsi2(j))/(GkRss(j)+b)));
99  -	        %Algorithme BNLMS
100 -	        w2(j)=(w(j)+((mu*es(j))/(Rss(j)+b)));
101 -	        
102 -	        %Réinitialisation des coefficients du filtre par BPNLMS++
103 -	        %Coefficients impairs par BPNLMS
104 -	        if mod(j,2)==1
105 -	            w(j)=w1(j);
106 -	        %Coefficients pairs par BNLMS    
107 -	        elseif mod(j,2)==0
108 -	            w(j)=w2(j);
109 -	        end
110 -	    end
111 -	end
112	-	w=((ones(1,64))*(13.5)-(w))/(13.5);
113 -	%Fin du chronométrage
114 -	toc;
\end{verbatim}
\textbf{Revenir dans le texte en cliquant \ref{Bloc}}.